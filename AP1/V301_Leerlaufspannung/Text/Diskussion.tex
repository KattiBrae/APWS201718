Initial lässt sich sagen, dass der Versuch keine großen Abweichungen zu dem erwarteten Ergebnissen aufweist.
\\Für die erste Messung, in der der Innenwiderstand $R_{i}$ und die Leerlaufspannung $U_{0}$ der Monozelle gemessen wird, ergeben sich folgende Messwerte:
\begin{align*}
  U_{0, Mono., theo} &= \SI{1.4}{V}\\
  R_{i} &= \SI{5.05722 \pm 0.00003}{\symup\Omega}\\
  U_{0, Mono1, exp} &= \SI{1.34431 \pm 0.00006}{V}.
\end{align*}
\\Die Abweichung zwischen der initial gemessenen Leerlaufspannung und der Leerlaufspannung aus Abbildung \ref{fig:Mono} sind \SI{4.14}\%.
\\Der systematische Fehler der initialen $U_{0}$-Messung ergibt sich zu
\begin{equation*}
  \Delta U_{0}=\frac{R_{i}}{R_{V}}=\SI{7.080108e-7}{V}.
\end{equation*}
\\Die Messung der Monozelle mit der Gegenspannungsmethode liefert diesen Innenwiderstand und Leerlaufspannung:
\begin{align*}
  U_{0, Mono., theo} &= \SI{1.4}{V}\\
  R_{i} &= \SI{529.614 \pm 0.101}{\symup\Omega}\\
  U_{0, Mono2, exp} &= \SI{1.443676 \pm 0.000113}{V}.
\end{align*}
\\Die Abweichung der theoretischen Leerlaufspannung und der aus Abbildung \ref{fig:Gegen} entnommenen Leerlaufspannung liegt bei \SI{3.12}\%.
\\Die Ergebnisse der Messung mit der angelegten Rechteckspannung lassen sich zusammenfassen zu:
\begin{align*}
  U_{0, Rechteck, theo} &=\SI{0.5}{V}\\
  R_{i} &= \SI{58.67 \pm 0.002}{\symup\Omega}\\
  U_{0, Rechteck, exp} &=\SI{0.497020 \pm 0.000018}{V}.
\end{align*}
\\Die Abweichung der theoretischen und der experimentellen Leerlaufspannung beläuft sich auf \SI{0.60}\%
\\Die Ergebisse der Messung zur angelegten Sinusspannung sind:
\begin{align*}
  U_{0, Sinus, theo} &=\SI{0.67}{V}\\
  R_{i} &= \SI{690.29 \pm 0.05}{\symup\Omega}\\
  U_{0, Sinus, exp} &=\SI{0.658594 \pm 0.000004}{V}.
\end{align*}
\\Die Abweichung der initial gemessenen theoretischen Leerlaufspannung und der experimentellen Leerlaufspannung aus der Abbildung \ref{fig:Sin} ergibt sich zu \SI{1.73}\%.
\\Die geringen Abweichungen lassen sich zum Beispiel durch parallaxe Fehler, prozentuale Messfehler der Geräte und durch systematische Fehler erklären.
Der systematische Fehler durch das hochohmige Voltmeter wird nur für die Monozelle ausgerechnet, aber ähnliche Fehler sind in allen Messreihen vorhanden.
\\Die gemessene Leistung $N(R_{a})$ in Abbildung \ref{fig:Leistung} weicht systematisch von der Theoriekurve ab.
Die systematische Abweichung lässt sich unter anderem durch den Fehler der Leerlaufspannung erklären.


\FloatBarrier
