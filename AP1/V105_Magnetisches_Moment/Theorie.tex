\subsection{Allgemeine Theorie}
Die kleinste Einheit, die ein magnetisches Moment erzeugen kann, ist ein magnetischer Dipol.
Magnetische Monopole gibt es nicht.
Magnetische Dipole können Permanentmagneten oder von Strom I durchflossene Leiter sein.
Letztere haben ein magnetisches Moment mit der Fläche A
\begin{equation*}
    \vec{\mu}= I \cdot \vec{A}.
\end{equation*}
Auch Permanentmagneten haben ein magnetisches Moment, welches aber nicht so einfach zu berechnen ist, sondern wie in diesem Versuch gemessen werden muss.
Mithilfe von geeigneten Spulen oder sehr großen Permanentmagneten lässt sich ein homogenes Magnetfeld schaffen.
Geeignete Spulen können hier sogenannte Helmholtzspulen sein.
Das sind zwei große, hintereinander stehende, ringförmige Spulen, zwischen denen sich ein nahezu homogenes Magnetfeld bildet.
Das Magnetfeld in der Mitte der beiden Spulen berechnet sich aus der magnetischen Feldkonstante \emph{$\mu_0$}, dem Radius R der Helmholtzspulen und dem Abstand d zwischen den beiden Spulen.
  \begin{equation}
    B=\frac{\mu_0 \cdot I \cdot R^2}{(R^2 + (\frac{d}{2})^2)^{3/2}}.
  \label{eqn:bfeld}
  \end{equation}

\subsection{Bestimmung des magnetischen Momentes unter Ausnutzung der Gravitation}
Wenn eine Masse m über einen Arm r an einen Drehpunkt befestigt wird, dann bewirkt die Gravitation g ein Drehmoment $D_{g}$
\begin{equation*}
  \vec{D}_{g}=m \cdot (\vec{r} \times \vec{g}).
\end{equation*}
Ein Magnet in einem homogenen Magnetfeld hat das Drehmoment $D_{B}$
\begin{equation*}
  \vec{D}_{B}=\vec{\mu}_{Dipol} \times \vec{B}.
\end{equation*}
Dabei ist $\mu$ das magnetische Moment und B die Magnetfeldstärke.
\\Diese Methode nutzt aus, dass das Drehmoment $D_{g}$ über die Gravitation und das magnetische Drehmoment $D_{B}$ Magneten im Gleichgewicht liegen.
Dabei haben $\vec{g}$ und $\vec{B}$ die selbe Richtung und die Äquivalenz kann beschreiben werden als:
  \begin{equation}
    \vec{\mu}_{Dipol}= m \cdot g \frac{r}{B}.
  \label{eqn:bmom1}
  \end{equation}

\subsection{Bestimmung des magnetischen Momentes über die Schwingungsdauer eines Magneten}
Die Schwingung eines Magneten in einem magnetischen Feld kann durch die Differentialgleichung eines harmonischen Oszillators beschrieben werden als:
\begin{equation*}
  -|\vec{\mu}_{Dipol} \times \vec{B}| = J_K \frac{d^2 \theta}{dt^2}.
\end{equation*}
Hier ist $J_K$ das Trägheitsmoment des Körpers und $\theta$ der Winkel.
\\Eine Lösung der Differentialgleichung liefert
  \begin{equation}
    \vec{\mu}_{Dipol}= \frac{4 \cdot \pi^2 \cdot J_K}{B \cdot T^2}.
  \label{eqn:bmom2}
  \end{equation}

\subsection{Bestimmung des magnetischen Momentes über die Präzession einer Kugel}
Eine Präzessionsbewegung entsteht, wenn die Drehachse eines rotierenden Körpers um eine andere, zweite Achse rotiert.
Die Präzession ist abhängig vom Drehimpuls $L_K$, der sich durch
\begin{equation}
 L_k = J_k \cdot 2 \pi \nu.
\label{eqn:lk}
\end{equation}
\\berechnen lässt. Der Drehimpuls ist wiederum abhängig von der Frequenz $\nu$.
\\Das Magnetfeld wirkt als äußere Kraft auf die Präzessionsbewegung, die durch die Differentialgleichung
\begin{equation*}
  \vec{\mu}_{Dipol} \times \vec{B} = \frac{d \vec{L}_{K}}{dt}
\end{equation*}
beschrieben werden kann. Eine Lösung der Differentialgleichung ist
\begin{equation*}
  \Omega_p = \frac{\mu B}{|\vec{L_K}|}.
\end{equation*}
\\Mit der Formel für das Trägheitsmoment und der Kreisfrequenz folgt
  \begin{equation}
    \vec{\mu}_{Dipol} = \frac{2 \cdot \pi \cdot L_K}{B \cdot T_p},
  \label{eqn:bmom3}
  \end{equation}
\\wobei $T_p$ die Umlaufdauer der Präzessionsbewegung ist.
