In diesem Experiment geht es darum das magnetische Moment eines Permanentmagneten durch drei verschiedene Methoden zu messen.
\begin{align*}
  \vec{\mu}_{\text{Gravitation}} &=& \SI{0.40753 \pm 0.00109}{A m^2}\\
  \vec{\mu}_{\text{Schwingung}}  &=& \SI{0.4149457 \pm 0.0000001}{A m^2}\\
  \vec{\mu}_{\text{Präzession}}  &=& \SI{0.42515559 \pm 0.000000009}{A m^2}\\
\end{align*}
Die relative Abweichung berechnet sich über
\begin{equation*}
  f_{1,2}=\frac{x_{1}-x_{2}}{x_{2}}.
\end{equation*}
\begin{align*}
  f_{\text{Grav,Schwing}} &=& \SI{1.79}{\%}\\
  f_{\text{Grav,Präz}}    &=& \SI{4.15}{\%}\\
  f_{\text{Schwing,Präz}} &=& \SI{2.40}{\%}\\
\end{align*}
\\Die Abweichung der drei gemessenen Ergebnisse kann verschiedene Gründe haben.
\\Bei der Bestimmung des magnetischen Momentes unter Ausnutzung der Gravitation kommen mehrere Probleme zusammen.
Der Aluminiumstift war bereits vor Beginn des Versuchs verbogen.
Ablesefehler sind nicht auszuschließen.
\\Die Bestimmung des magnetischen Momentes über die Schwingungsdauer des Magneten ist ebenfalls nicht ohne Fehlerquellen.
Messfehler sind auch bei der Schwingungszahl und der Schwingungsdauer möglich.
Die Auslenkung der Billardkugel ist von Messung zu Messung nicht konstant und möglicherweise zu groß.
\\Die Bestimmung des magnetischen Momentes über die Präzession der Billardkugel bringt die meisten Fehlerquellen mit sich.
Es wird beobachtet, dass die Billardkugel sich initial, bereits vor der Auslenkung, mit einer geringen Präzession dreht.
Der Zeitpunkt, an dem die Kugel die richtige Frequenz hat, ist schwierig zu erkennen.
Im Laufe der Messungen wird es auch immer schwieriger, da das Stoboskoplicht sehr anstrengend für die Augen ist.
Es ist nicht auschließbar, dass es bei dem Starten der Uhr und bei dem Aufdrehen der Spulenstromstärke durch Verzögerungen zu fehlerhaften Werten kommt.
Desweiteren gestaltete es sich schwierig, den genauen Zeitpunkt der Umlaufdauer der Präzessionsbewegung zu messen, da die Präzession bei den Messungen zum Teil recht unterschiedliche Auslenkungswinkel hat.
