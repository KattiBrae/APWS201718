Initial ist auffällig, dass die theoretischen Werte generell größer als die experimentellen Werte sind.
\\Der experimentelle Wert des effektiven Dämpfungswiderstands $R_{eff}$ kann nicht gut verglichen werden, da kein genauer theoretischer Wert gegeben ist.
Es kann aber davon ausgegangen werden, dass zu dem Widerstand R ein Innenwiderstand des Oszilloskops und ein Innenwiderstand des Generators hinzukommt.
\\Der theoretische Wert der Abklingdauer $T_{ex}$ ist 140,51\% größer als der experimentell gemessene Wert.
\\Der theoretische Dämpfungswiderstand $R_{ap}$ ist 26,52\% größer als der gemessene Wert.
\\Die theoretische Resonanzüberhöhung q ist 160,78\% größer als der Wert aus der Messung.
\\Der theoretische Wert der Halbwertsbreite b ist 30,6\% größer als der experimentelle.
\\Die theoretische Resonanzfrequenz $\nu_{res}$ ist 0,29\% größer als der gemessene Wert der Resonanzfrequenz.
\\Der Wert der Frequenz $\nu_{1}$ ist theoretisch errechnet 4,25\% größer als der gemessene Wert.
\\Der theoretische Wert für $\nu_{2}$ ist 6,23\% größer als der aus der Messung abgelesene Wert.
Die Abweichungen lassen sich unter anderem damit erklären, dass die Innenwiderstände der Geräte nicht beachtet wurden.
Da alle theoretischen Ergebnisse direkt oder indirekt davon betroffen sind, zieht sich die Ungenauigeit durch alle theoretischen Werte.
Außerdem können systematische Ablesefehler mit trotz Nutzung der Cursorfunktion des Oszilloskops nicht ausgeschlossen werden.
\\Dennoch entsprechen die Verläufe der Messkurven den erwarteten Kurven.
