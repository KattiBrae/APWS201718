\subsection{Gedämpfte Schwingungen}
In einem Stromkreis mit einem Kondensator der Kapazität C und einer Spule der Induktivität L kann es zu ungedämpften Schwingungen des Stroms I, der Spannung U und der Energie kommen.
Wird ein Widerstand R in den Stromkreis eingesetzt, werden aus den ungedämpften Schwingungen gedämpfte Schwingungen.
Der Widerstand entfernt fortlaufend mit der Zeit Energie aus dem Stromkreis und die Amplituden der Schwingungen nehmen ab, bis sie nach einer gewissen Zeit gänzlich abgeklungen sind.
\\Das zweite Kirchhoffsche Gesetz ergibt für diesen Schaltkreis folgende Gleichung:
\begin{equation}
  U_{R}(t) + U_{C}(t) + U_{L}(t) = 0.
  \label{eqn:kirch2}
\end{equation}
\\Mit den Beziehungen
\begin{equation*}
  U_{R}(t) = R \cdot I(t),
\end{equation*}
\begin{equation*}
  U_{C}(t) = \frac{Q(t)}{C},
\end{equation*}
\begin{equation*}
  U_{L}(t) = L \cdot \frac{dI}{dt}
\end{equation*}
\begin{equation*}
  I = \frac{dQ}{dt}
\end{equation*}
\\wird aus Formel \eqref{eqn:kirch2}:
\begin{equation}
  L \cdot \frac{dI}{dt} + R \cdot I + \frac{Q}{C} = 0.
  \label{eqn:DGL1}
\end{equation}
\\Dabei ist Q die Ladung.
\\Die DGL \eqref{eqn:DGL1} wird nun zeitlich abgeleitet und man erhält die DGL der gedämpften Schwingung:
\begin{equation}
  \frac{d^2 I}{d t^2} + \frac{R}{L} \frac{dI}{dt} + \frac{1}{LC} I = 0.
\label{eqn:DGLgedämpft}
\end{equation}
\\Die Lösung der DGL \eqref{eqn:DGLgedämpft} lautet:
\begin{equation}
  I(t)= e^{-2 \pi \mu t} \left( U_{1} e^{i 2 \pi \nu t} + U_{2} e^{-i 2 \pi \nu t} \right)
  \label{eqn:lsg}
\end{equation}
\\mit den Substitutionen
\begin{equation}
  \mu = \frac{R}{4 \pi L} \Leftrightarrow R_{eff}= 4 \pi L \mu
  \label{eqn:mu}
\end{equation}
\begin{equation*}
  \nu = \frac{1}{2 \pi} \sqrt{\frac{1}{LC} - \frac{R^2}{4 L^2}}.
\end{equation*}
\\Für $1/LC > R^2/4L^2$ ist \nu reell und aus Formel \eqref{eqn:lsg} ergibt sich:
\begin{equation*}
  I(t)= A_{0} \cdot e^{-2 \pi \mu t} \cdot \cos{(2 \pi \nu t + n)}.
\end{equation*}
\\Dazu wird die Abklingdauer
\begin{equation}
  T_{ex} = \frac{1}{2 \pi \mu} = \frac{2L}{R}
  \label{eqn:tex}
\end{equation}
\\definiert, nach der die Amplitude auf den $1/e$-ten Teil abgesunken ist.
\\Für $1/LC < R^2/4L^2$ ist \nu komplexwertig und aus Formel \eqref{eqn:lsg} ergibt sich:
\begin{equation*}
  I(t) \propto e^{-(2 \pi \mu - i 2 \pi \nu) t}.
\end{equation*}
\\Der aperiodische Grenzfall ist ein wichtiger Spezialfall.
Dabei gilt
\begin{equation}
  \frac{1}{LC} = \frac{R_{ap}^2}{4 L^2} \Leftrightarrow R_{ap} = 2 \sqrt{\frac{L}{C}}
  \label{eqn:rap}
\end{equation}
\\mit $\nu=0$.

\subsection{Erzwungene Schwingungen}
Der zuvor beschriebene Stromkreis wird nun um eine Erregerspannung
\begin{equation}
  U(t) = U_{0} e^{i \omega t}
\end{equation}
\\erweitert. Dabei ist \omega die Kreisfrequenz oder auch Winkelgeschwindigkeit und $U_{0}$ ist die Amplitude der Erregerspannung.
\\Die DGL \eqref{eqn:DGL1} wird nun zu
\begin{equation}
    LC \cdot \frac{d^2 U_{C}}{dt^2} + RC \cdot \frac{d U_{C}}{dt} + U_{C} = U_{0} e^{i \omega t}.
  \label{eqn:DGLerzwungen}
\end{equation}
\\Die Lösung der DGL \eqref{eqn:DGLerzwungen} lautet:
\begin{equation*}
  U(t) = \frac{U_{0}(1-LC \omega^2 - i \omega RC)}{(1- LC \omega^2)^2 + \omega^2 R^2 C^2}.
\end{equation*}
\\Die Phasenverschiebung \varphi kann dann durch
\begin{equation}
  \varphi (\omega) = \arctan{ \left( \frac{- \omega R C}{1- \omega^2 L C} \right) }
  \label{eqn:varphi}
\end{equation}
\\beschrieben werden.
\\Die Kondensatorspannung $U_{C}$ in Abhängigkeit der Winkelgeschwindigkeit \omega wird auch Resonanzkurve genannt und kann wie folgt angegeben werden:
\begin{equation*}
  U_{C}(\omega)= \frac{U_{0}}{\sqrt{\left( 1-\omega^2 LC \right)^2 + \omega^2 R^2 C^2}}.
\end{equation*}
\\Unter Beachtung der Beziehung $\omega=2 \pi \nu$ lautet die Resonanzfrequenz $\nu_{res}$ dann
\begin{equation}
  \nu_{res}= \frac{1}{2 \pi} \sqrt{\frac{1}{LC} - \frac{R^2}{2L^2} }.
  \label{eqn:nuresonanz}
\end{equation}
\\Bei schwachen Dämpfungen gibt es eine auffällige Resonanzüberhöhung, die bis zur Resonanzkatastrophe reichen kann.
Die Resonanzüberhöhung q wird durch die Halbwertsbreite der Resonanzkurve charakterisiert.
Die Halbwertsbreite b ist hier ist der Abstand der Werte an denen die Funktion auf den $1/\sqrt{2}$-ten Teil des Maximums gewachsen bzw. abgefallen ist:
\begin{equation}
  b = \omega_{+} - \omega_{-} = \frac{R}{L}.
  \label{eqn:halbwertsbreiten}
\end{equation}
\\Die Resonanzüberhöhung wird durch die folgende Gleichung beschrieben:
\begin{equation}
  q = \frac{1}{\omega_{0}RC} = \frac{\sqrt{L}}{R \sqrt{C}}.
  \label{eqn:q}
\end{equation}
\\Aus Gleichung \eqref{eqn:varphi} kann für eine Verschiebung von $\pi/4$ und $3\pi/4$ folgende Beziehung entnommen werden:
\begin{equation}
  \nu_{1,2} = \pm \frac{R}{4 \pi L} + \frac{1}{2 \pi} \sqrt{\frac{R^2}{4 L^2} + \frac{1}{LC} }.
  \label{eqn:phasenverschiebung}
\end{equation}
