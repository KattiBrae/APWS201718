Auffällig ist, dass die Abweichung der experimentellen Werte zu den theoretischen Werten in diesem Experiment gering ist.
Die größte prozentuale Abweichung im gesamten Experiment liegt bei 0,0138\% und beschreibt die Abweichung der theoretischen und der experimentellen Amplitude der neunten Oberwelle der Dreieckspannung.
Die weiteren Verhältnisse sind in den Tabellen \ref{tab:recht}, \ref{tab:drei} und \ref{tab:säg} angegeben.
Durch den Versuchsaufbau sind wenige Messunsicherheiten gegeben.
Das Oszilloskop, der Funktionsgenerator, der Oberwellengenerator und das Voltmeter sind genaue digitale Geräte, die nur äußerst geringe Messunsicherheiten verursachen.
