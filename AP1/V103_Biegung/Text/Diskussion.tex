Der eckige Stab und der runde Stab können beide als Messinglegierung bestimmt werden.
Der Literaturwert für Messing \cite{1} wird als 8100 bis 8700 $\frac{\symup{kg}}{\symup{m^3}}$ angegeben.
Beide Dichten der Stäbe liegen mit
\begin{align*}
  \rho_{e} &=& \SI{8175.93359 \pm 0.14050}{\frac{kg}{m^3}}\\
  \rho_{r} &=& \SI{8353.8582 \pm 0.0019}{\frac{kg}{m^3}}.\\
\end{align*}
innerhalb dieses Bereiches.

Der verwendete Literaturwert \cite{2} für das Elastizitätsmodul von Messing liegt bei
\begin{align*}
              &&&  E_{Lit,u} &=& \SI{ 78}{\frac{kN}{mm^2}} &=& \SI{ 78e09}{\frac{N}{m^2}} \\
   \text{bis} &&&  E_{Lit,o} &=& \SI{123}{\frac{kN}{mm^2}} &=& \SI{123e09}{\frac{N}{m^2}}.\\
\end{align*}
Die errechneten Elastizitätsmodule ergeben sich zu:
\begin{align*}
  E_{e}  &=&  \SI{139.51543 \pm 2.31330e09}{\frac{N}{m^2}}\\
  E_{r}  &=&  \SI{101.45513 \pm 0.19260e09}{\frac{N}{m^2}}\\
  E_{br} &=&  \SI{109.60296 \pm 1.81732e09}{\frac{N}{m^2}}\\
  E_{bl} &=&  \SI{241.32550 \pm 4.00141e09}{\frac{N}{m^2}}.\\
\end{align*}
Zum Vergleich der Werte wird die relative Abweichung $f$ berechnet:
\begin{equation*}
  f_{\text{a,b}}=\frac{x_{\text{a}}-x_{\text{b}}}{x_{\text{b}}}.
\end{equation*}
\begin{align*}
  f_{\text{e,Lit,o}}  &=& \SI{13.43}{\%}  &&&& \text{Eckiger Stab (einseitig), oberer Literaturwert}\\
%  f_{\text{r,Lit,o}}  &=& \SI{17.52}{\%}  &&&& \text{Runder Stab (einseitig), oberer Literaturwert}\\
%  f_{\text{r,Lit,u}}  &=& \SI{30.07}{\%}  &&&& \text{Runder Stab (einseitig), unterer Literaturwert}\\
  f_{\text{bl,Lit,o}} &=& \SI{96.20}{\%}  &&&& \text{Eckiger Stab (beidseitig), oberer Literaturwert}\\
  f_{\text{e,r}}      &=& \SI{37.51}{\%}  &&&& \text{Eckiger Stab (einseitig), runder Stab (einseitig)}\\
  f_{\text{e,br}}     &=& \SI{27.29}{\%}  &&&& \text{Eckiger Stab (einseitig), eckiger Stab (beidseitig, rechts)}\\
  f_{\text{e,bl}}     &=& \SI{42.19}{\%}  &&&& \text{Eckiger Stab (einseitig), eckiger Stab (beidseitig, links)}\\
  f_{\text{br,bl}}    &=& \SI{54.58}{\%}  &&&& \text{Eckiger Stab (beidseitig, rechts), eckiger Stab (beidseitig, links)}\\
\end{align*}
Der Messwert des Elastizitätsmoduls des einseitig eingespannten eckigen Stabs und des beidseitig eingespannten eckigen Stabs ($\frac{l_{e}}{2} \geq x \geq 0$) liegen deutlich oberhalb der Grenze des oberen Literaturwerts.
Im Gegensatz dazu sind die Elastizitätsmodule des einseitig eingespannten runden Stabs und des beidseitig eingespannten eckigen Stabs ($l_{e} \geq x \geq \frac{l_{e}}{2}$) in den Grenzen des Literaturwerts.
Die Elastizitätsmodule des beidseitig eingespannten Stabs weichen stark voneinander ab.
\\Die Gründe für diese Abweichungen können vielfältig sein.
Zunächst ist die Messunsicherheit einer Schieblehre recht groß, im Verhältnis zu anderen Messmethoden.
Mit der Schieblehre werden die Stäbe vermessen, entsprechend könnte hier ein falsches Element bestimmt worden sein.
Desweiteren hat die Messuhr eine Messunsicherheit.
Zwischenzeitlich kann beobachtet werden, wie der Zeiger ohne Einwirkung um $x=\SI{0.2}{mm}$ vor- oder zurückspring und äußerst sensibel auf Erschütterungen reagiert.
Die Messuhr springt auch bei dem Zurückschieben auf den ersten x-Wert nicht wieder auf 0.
Es ist nicht auszuschließen, dass die Messuhr ungenau verschoben wird und die x-Werte entsprechend verfälscht sind.
Ebenso ist es möglich, dass das Abwiegen der Massen der Stäbe und der Gewichte nicht genau ist.
