\section{Zielsetzung}
Ziel des Versuches ist es das Trägheitsmoment unterschiedlicher Körper zu ermitteln.
Zusätzlich wird der Satz von Steiner verifiziert.


\section{Theorie}
Rotationsbewegungen lassen sich durch das Drehmoment M, die Winkelbeschleunigung $\dot{\omega} $ und das Trägheitsmoment I beschreiben.
Das Trägheitsmoment beschreibt die Massenverteilung eines Körpers im Bezug auf die Rotationsachse.
Das Gesamtträgheitsmoment lässt sich dabei beschreiben durch

\begin{equation*}
  I = \sum_i r_i^2 \cdot m_i.
\end{equation*}
\\$m_i$ ist dabei ein Massenelement mit dem Abstand $r_i$ zur Drehachse. Für infinitesimale Körper ergibt sich

\begin{equation*}
  I = \int r^2 dm.
\end{equation*}
\\Geht die Drehachse nicht durch den Schwerpunkt des Körpers muss das Trägheitsmoment durch den Satz von Steiner beschrieben werden:

\begin{equation}
  I = I_s + m \cdot a^2.
  \label{eqn:steiner}
\end{equation}
\\$I_s$ ist dabei das Tragheitsmoment bezüglich der Drehachse des Schwerpunktes.
Für das Drehmoment M eines Körpers auf den die Kraft F im Abstand r wirkt, gilt die Formel:

\begin{equation*}
  \vec{M} = \vec{F}\times \vec{r}.
\end{equation*}
\\Stehen F und r senkrecht zueinander gilt

\begin{equation*}
  M = F\cdot r.
\end{equation*}
\\Bei schwingenden System wirkt der Auslenkung $\phi$ ein rücktreibendes Drehmoment entgegen, welches beschrieben wird durch:

\begin{equation*}
  M = D \cdot \phi
\end{equation*}
\\mit der Winkelrichtgröße D.
\\Die Schwingungsdauer beträgt

\begin{equation}
  T = 2\pi\sqrt\frac{I}{D}.
\label{eqn:schwing}
\end{equation}
\\Die Formel gilt jedoch nur für kleine Winkel. Sie lässt sich umstellen zu
\begin{equation}
  I = \frac{T^2 \cdot D}{(2 \pi)^2}.
\label{eqn:schwing2}
\end{equation}
\\Das Trägheitsmoment I eines Zylinders, bei dem die Drehachse durch den Schwerpunkt geht und senkrecht zur Bodenfläche steht, ist das Trägheitsmoment
\begin{equation}
  I_{zs} = \frac{m \cdot R^2}{2}.
\label{eqn:Zs}
\end{equation}
\\Für einen liegenden Zylinder, bei dem die Achse durch den Schwerpunkt geht, ist das Trägheitsmoment
\begin{equation}
  I_{zl} = m \left(\frac {R^2}{4}+\frac {h^2}{12} \right).
\label{eqn:Zl}
\end{equation}
\\Ein Stab, der sich um eines seiner Enden dreht, hat das Trägheitsmoment
\begin{equation}
  I_{st} = \frac{m \cdot L^2}{3}
  \label{eqn:stab}
\end{equation}
