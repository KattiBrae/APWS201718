Auf einer zweifach eingespannten, drehbaren Achse können verschiedene Körper befestigt werden.
Die Achse ist über eine Feder mit dem Rahmen verbunden.
Zur Berechnung späterer Trägheitsmomente werden die Konstanten der Feder benötigt.
\\D wird bestimmt, indem eine Federwaage an einem bestimmten Abstand zur Achse an eine nahezu masselosen Stange ansetzt wird und diese Stange den Winkel $\phi$ ausgelenkt wird.
An der Federwaage ist die zugehörige Kraft ablesbar.
Es wird die Kraft für 10 Winkel gemessen.
\\Das Eigenträgheitsmoment $I_D$ wird bestimmt, indem eine Stange mit zwei zylinderförmigen Gewichten senkrecht zur Drillachse befestigt wird und diese in Schwingung versetzt wird.
Die Schwingungsdauer ist für 10 unterschiedliche Abstände der Massen zur Drehachse zu messen.
\\Anschließend wird das Trägheitsmoment von einem liegenden und einem stehenden Zylinder ermittelt.
Dafür wird wieder die Schwingungsdauer der Körper gemessen.
\\Eine Holzpuppe wird nach der Vermessung ihrer Proportionen und nach Messen des Gewichtes, auf die Drillachse gesteckt.
Es werden für zwei verschiedenen Körperhaltungen die Schwingungszeiten ermittelt.
Zunächst werden die Schwingungsdauern der Puppe mit angelegten Armen gemessen, dann werden die Schwingungsdauern der selben Puppe mit seitlich ausgestreckten Armen gemessen.
