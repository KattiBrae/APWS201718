Initial fällt auf, dass die experimentellen Werte alle kleiner als die theoretischen Werte sind.
Der theoretische Wert für das Trägheitsmoment des ersten Körpers ist 2236,82\% kleiner als der Experimentelle.
Das Trägheitsmoment des zweiten Körpers ist über die theoretischen Werte 1958,72\% kleiner als aus der experimentellen Messung.
Der gemessene, experimentelle Wert für das Trägheitsmoment der Puppe in der ersten Körperhaltung ist 1466,94\% größer als der theoretische Wert.
Der experimentelle Wert des Trägheitsmomentes der Puppe in der zweiten Körperhaltung ist 1453.83\% größer als der Wert aus der Modellrechnung.
\\Für die Abweichungen sind einige Gründe möglich. Zunächst können Fehler bei dem Wiegen der Massen, bei dem Ablesen des Kraftmessers,
bei dem Messen der Schwingungsdauern und bei dem Ablesen der Abstände auf der Schieblehre nicht ausgeschlossen werden.
Es ist bekannt, dass das Verwenden einer Schieblehre häufig zu systematischen Fehlern führt.
\\In dem Experiment fällt außerdem schnell auf, dass die Schwingungsdauern der einzelnen Körper zum Teil sehr kurz waren und äußerst schwierig manuell zu messen.
Ebenfalls auffällig ist der negative experimentelle Wert für das Eigenträgheitsmoment der Drillachse. Ein negatives Trägheitsmoment ist physikalisch nicht möglich.
Dies legt nahe, dass ein geringes Eigenträgheitsmoment der Drillachse vorhanden ist, die Messfehler aber zu einem negativen Wert geführt haben.
\\Die Erwartung, dass das Trägheitsmoment einer Puppe mit angelegten Armen kleiner ist als das Trägheitsmoment einer Puppe mit ausgestreckten Armen wurde erfüllt.
