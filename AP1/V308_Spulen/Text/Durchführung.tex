\subsection{Magnetfelder verschiedener Spulen}
\subsubsection{Lange Spule}
Als erstes wird eine längliche Spule mit der Windungszahl $N=300$ und der Länge $l=\SI{0.18}{m}$ an ein Netzgerät angeschlossen.
Der magnetfelderzeugende Strom $I=\SI{1}{A}$ wird eingeregelt und mit einer longitudinalen Hall-Sonde wird das Magnetfeld entlang der Längsachse gemessen.
Die Messwerte innerhalb der Spule werden aufgenommen, bis die Sonde nicht weiter in die Spule versenkt werden kann.
Es werden Messwerte ab $\SI{0.04}{m}$ vor der Spule aufgenommen.

\subsubsection{Kurze Spule}
Die kurze Spule mit der Windungszahl $N=300$ und der Länge $l=\SI{0.08}{m}$ wird an das Netzgerät geschlossen.
Der Strom wird auf $I=\SI{1}{A}$ eingestellt und die Messung wird analog zur Messung an der langen Spule durchgeführt.
Die Messwerte werden ab $\SI{0.03}{m}$ vor der Spule aufgenommen.

\subsection{Magnetfelder einer Helmholtzspule}
Die Helmholtzspulen mit der jeweiligen der Windungszahl $N=100$ und dem Spulenradius $R=\SI{0.0625}{m}$ werden an das Netzgerät geschlossen.
Der Strom wird auf $I=\SI{1}{A}$ eingestellt.
Der Spulenabstand ist variabel wird zunächst auf $d_{1}=\SI{0.08}{m}$ eingestellt.
Das Magnetfeld wird zuerst mit der transversalen Hall-Sonde entlang der Verbindungslinie der Mittelpunkte der Spulen ausgemessen, danach wird auch entlang der Verlängerung der Verbindungslinie gemessen.
Die Messung wird für die Spulenabstände $d_{2}=\SI{0.10}{m}$ und $d_{3}=\SI{0.12}{m}$ wiederholt.

\subsection{Hysteresekurve}
Zur Messung der Hysteresekurve wird die Toroidspule mit der Windungszahl $N=595$ und einem Luftspalt der Breite $b=\SI{0.003}{m}$ an das Netzgerät angeschlossen.
Die transversale Hall-Sonde wird in dem Luftspalt angebracht.
\\Der magnetfelderzeugende Strom wird Schrittweise auf $I=\SI{10}{A}$ erhöht um die Neukurve zu messen.
Danach wird der Strom von $\SI{10}{A}$ auf $\SI{0}{A}$ reduziert.
An dieser Stelle wird umgepolt.
Dann wird der Strom wieder von $\SI{0}{A}$ auf $\SI{10}{A}$ erhöht.
Der Strom wird von $\SI{10}{A}$ auf $\SI{0}{A}$ verringert.
Es wird umgepolt und der Strom wird wieder auf $\SI{10}{A}$ erhöht.
Damit ist die Hysteresekurve ausgemessen.
