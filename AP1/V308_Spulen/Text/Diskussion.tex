Allgemein fällt auf, dass die gemessenen Größen und die Verläufe den Erwartungen entsprechen.
\\In der langen Spule ist das B-Feld nach einer gewissen Eindringtiefe konstant und ergibt ein Plateau im Graphen.
Die Abweichung von dem theoretischen Wert
\begin{equation*}
  B_{LS, t} = \SI{2.094395e-3}{T}
\end{equation*}
\\zu dem aus dem Graphen abgelesenen experimentellen Wert
\begin{equation*}
  B_{LS, e} = \SI{2.160e-3}{T}
\end{equation*}
\\des Magnetfelds in der Spule ist 3,13\%.
\\Die kurze Spule hat kein homogenes Magnetfeld sondern nur ein Maximum des B-Felds auf Höhe der Mitte der Spule.
Es kann als parabolisches Profil beschrieben werden.
Der theoretischen Wert des Magnetfelds in der Mitte der kurzen Spule kann nur genähert werden:
\begin{equation*}
  B_{KS, t} = \SI{4.712389e-3}{T}.
\end{equation*}
\\Für den aus dem Graphen abgelesene experimentelle Wert ergibt sich:
\begin{equation*}
  B_{KS, e} = \SI{1.885e-3}{T}
\end{equation*}
\\Die Abweichung beläuft sich auf 149,99\%.
Die große Abweichung lässt sich unter anderem dadurch erklären, dass für den Theoriewert die Formel einer langen Spule verwendet wird.
Diese Formel beschreibt das B-Feld einer kurzen Spule nur mäßig.
\\Bei den Helmholtzspulen ist erkennbar, wie wichtig der Spulenabstand ist, um das Magnetfeld zwischen den Spulen als homogen zu nähern.
In den Graphen ist eine Überlagerung zweier parabolischer Profile ersichtlich.
Je größer der Spulenabstand ist, desto deutlicher ist zu erkennen, dass sich die parabolischen Profile der Magnetfelder der beiden kurzen Spulen überlagern.
Der theoretische Wert für das B-Feld ist hier immer größer als der Experimentelle.
\\Der theoretische Wert zum Spulenabstand $d_{1}=\SI{0.08}{m}$ ist
\begin{equation*}
  B_{H1, t} = \SI{12.01395e-4}{T}.
\end{equation*}
\\Der experimentelle Wert ergibt sich zu
\begin{equation*}
  B_{H1, e} = \SI{11.97e-4}{T}.
\end{equation*}
\\Die Abweichung der beiden Werte beträgt 0,38\%.
\\Für den Spulenabstand $d_{2}=\SI{0.10}{m}$ wird ein theoretischer Wert von
\begin{equation*}
  B_{H2, t} = \SI{9.573353e-4}{T}
\end{equation*}
\\errechnet. Der zugehörige experimentelle Wert beläuft sich auf
\begin{equation*}
  B_{H2, e} = \SI{9.73e-4}{T}.
\end{equation*}
\\Die Abweichung wird als 1,64\% bestimmt.
\\Der theoretische Wert zum Spulenabstand $d_{3}=\SI{0.12}{m}$ wird als
\begin{equation*}
  B_{H3, t} = \SI{7.548060e-4}{T}
\end{equation*}
\\berechnet. Der experimentelle Wert ist
\begin{equation*}
  B_{H3, e} = \SI{7.76e-4}{T}.
\end{equation*}
\\Die Abweichung ist 2,81\%.
\\Bei der Toroidspule ist kein Theoriewert ermittelbar.
Außerdem kann das H-Feld nicht ausgerechnet werden, da der Radius der Toroidspule nicht bekannt ist.
Da das H-Feld direkt abhängig vom magnetfelderzeugenden Strom ist, kann gleichermaßen B gegen I aufgetragen werden.
Es ergibt sich eine Hysteresekurve.
Diese Kurve kann nun nicht mit Literaturwerten verglichen werden.
\\Generell lässt sich sagen, dass die Erwartungen erfüllt wurden.
