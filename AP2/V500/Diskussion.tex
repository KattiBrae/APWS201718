Initial lässt sich sagen, dass sich das Aufnehmen der Messwerte schwierig gestaltet.
Die Fehlerquellen sind so zum Beispiel defekte Kabel an der Anode und Kathode und starke Schwankungen des Photostroms ohne Veränderung der Parameter.
In der ersten Messreihe wird schnell festgestellt, dass sich bei der niederenergetischen orangenen Wellenlänge $\lambda=\SI{578}{nm}$ ohne Gegenspannung gar kein Photostrom einstellt.
Nach dem Wechseln der Messapparatur wird klar, dass die Intensität der ersten Spektrallampe nicht stark genug war, um einem Photostrom auszulösen.
Die Messung wird erneut gestartet.
\\Auch mit der zweiten Messapparatur kommen Probleme auf.
Bei den höherenergetischen Wellenlängen $\lambda=\SI{405}{nm}$ und $\lambda=\SI{436}{nm}$ (Violett) lassen sich die Messwerte ganz gut ablesen.
Dennoch lässt sich eine gewisse Schwankung des Messzeigers des Amperemeters ausmachen.
Besonders bei den niederenergetischeren Wellenlängen schwankt das Amperemeter beträchtlich, ohne dass die Parameter geändert werden.
Für das cyanfabene Licht ($\lambda=\SI{492}{nm}$) sind die Schwankungen besonders stark.
So fällt das Amperemeter in der Messung von etwa $I=\SI{0.040}{nA}$ ohne Änderung der Parameter auf $I=\SI{0.020}{nA}$ um dann auf $I=\SI{0.060}{nA}$ zu steigen.
Die Graphen zu den schwierig messbaren niederenergetischen Wellenlängen $\lambda=\SI{578}{nm}$ (Abb. \ref{fig:o578}), $\lambda=\SI{546}{nm}$ (Abb. \ref{fig:g546}) und $\lambda=\SI{492}{nm}$ (Abb. \ref{fig:g492}) haben dennoch den erwarteten linearen Verlauf.
Auch der Graph zur Wellenlänge $\lambda=\SI{436}{nm}$ (Abb. \ref{fig:v405}) liegt im erwarteten linearen Rahmen.
\\Im Gegensatz dazu scheint der Graph zur Wellenlänge $\lambda=\SI{405}{nm}$ (Abb. \ref{fig:v436}) ungenau.
\\Die Größe $h/e_0$ wird zu
\begin{equation*}
  \frac{h}{e_0} = \SI{2.457 \pm 0.060} 10^{-15}\si{\frac{A}{V}}
\end{equation*}
\\berechnet.
Der Literaturwert wird aus den Größen $h$ und $e_0$ berechnet \cite{taschenrechner}:
\begin{equation*}
  \frac{h}{e_0}=\SI{4.136e-15}{Js}
\end{equation*}
\\Die relative Abweichung $f$ berechnet sich durch:
\begin{equation*}
  f=\frac{x_{exp}-x_{theo}}{x_{theo}}
\end{equation*}
und beträgt $f=\SI{40.58}{\%}$.
Die Austrittsarbeit des Kathodenmaterials wird als
\begin{equation*}
  A_K=\SI{0.687 \pm 0.024}{eV}
\end{equation*}
\\bestimmt.
\\In der zweiten Messreihe wird der Photostromverlauf über eine breitere Gegenspannung abgemessen (Abb. \ref{fig:Messreihe2}).
Auffällig ist, dass bei sehr niedrigen Gegenspannungen ein Sättigungsstrom auftritt.
Dieser ist dadurch zu erklären, dass der Strom abhängig von der Intensität ist.
Da die Intensität der Spektrallampe konstant ist, kann nur ein maximaler Strom auftreten.
Ein Sättigungsstrom bei niedrigen Gegenspannungen ließe sich zum Beispiel durch eine geringere Intensität des Lichtes erzeugen.
Eine weitere Möglichkeit ist, bei gleicher Frequenz des Lichtes und gleicher Bremsspannung die Austrittsarbeit des Materials zu erhöhen, dies ist aber schwer zu realisieren.
Damit sind die Elektronen nach Austritt aus dem Material langsamer und der Strom wird kleiner.
\\Außerdem ist dem Graphen zu entnehmen, dass der Photostrom auch schon bei Gegenspannungen von $U > U_{g}$ sehr klein wird.
Rückblickend lässt sich sagen, dass weitere Messwerte im Bereich der Gegenspannungen von $U=0V$ bis $U=1V$ dieses Phänomen besser gezeigt hätten.
\\Desweiteren kommt auch ein negativer Photostrom zustande.
Diesem Phänomen liegt die niedrige Siedetemperatur des Kathodenmaterials zugrunde.
Bei Raumtemperatur verdampft bereits ein geringer Teil des Kathodenmaterials und lagert sich auf der Anode ab.
Durch Reflexionen gelangt auch Licht auf die Anode und Elektronen werden aus dem Anodenmaterial ausgeschlagen.
\\Der negative Photostrom bei niederenergetischen Wellenlängen lässt sich durch die niedrige Austrittsarbeit des Anodenmaterials erklären.
