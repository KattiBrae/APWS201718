Die $\alpha$-Strahlung entsteht durch den radioaktiven Zerfall eines Atomkerns. Bei dem ausgesendeten Teilchen handelt es sich um ein Helium Atomkern. Dieses besteht aus zwei Neutronen und zwei Protonen und ist somit zweifach positiv geladen.
Die $He$-Teilchen treten durch den Tunneleffekt aus dem Kernpotenzial des Atomkerns.
Das Kernpotenzial setzt sich zusammen aus dem Coulombpotential durch den Ladungsunterschied von Elektronen und Atomkern und aus den Kernkräften.
Die Einteilung der Strahlung in $\alpha, \beta,$ und $\gamma$ wurde über die Reichweite vorgenommen. Die $\alpha$-Strahlung ist demnach die Strahlung mit der geringsten Reichweite.
Die Reichweite eines $\alpha$-Teilchens wird durch Ionisations-, Anregungs-, und Dissoziotions-Prozesse bestimmt. Die Energieabgabe ist dabei Abhängig von der Dichte des Mediums und der Geschwindigkeit des $\alpha$-Teilchens.
Für große Energien kann der Energieverlust($-\frac{dE_{\alpha}}{dx}$) über die Bethe-Bloch-Gleichung bestimmt werden.
\begin{align*}
  -\frac{dE_{\alpha}}{dx}=\frac{z^2e^4}{4\pi\epsilon_0m_e}\frac{nZ}{v}ln\left(\frac{2m_ev^2}{I}\right)
\end{align*}
$z$ beschreibt die Ladung und $v$ die Geschwindigkeit der $\alpha$-Teilchen. Die Ordungszahl wird durch $Z$ ausgedrückt. $n$ ist die Teilchendichte und I dei Ionisationsenergie des Mediums.
Wegen Ladungsaustauschprozessen gilt die Bethe-Bloch-Gleichung nicht für sehr kleine Energien.
Nach vollständiger Energieabgabe hat das $\alpha$-Teilchen seine maximale Reichweite $R$ errecht. Diese Reichweite kann durch die Formel
\begin{align*}
  R=\int^{E_{\alpha}}_{0}\frac{dE_{\alpha}}{-\frac{dE_{\alpha}}{dx}}
\end{align*}
berechnet werden.
Wegen der Ungültigkeit der Bethe-Bloch-Gleichung für kleine Energien muss die mittlere Reichweite mit empirischen Kurven ermittelt werden.
Für $\alpha$-Teilchen mit einer Energie unter $\SI{2,5}{MeV}$ kann die Beziehung $Rm= 3,1 \cdot E_{\alpha}^{\frac{3}{2}}$ verwendet werden.
Bei konstanter Temperatur und bei konstantem Volumen ist die Reichweite der $\alpha$-Teilchen proportional zum Druck.
Variiert man nun den Druck bei konstanter Länge $x_0$ kann so durch eine Absorptionsmessung die effektive Länge bestimmt werden.
\begin{align}
  x=x_0\frac{p}{p_0}
  \label{eqn:x}
\end{align}
$p_0$ ist dabei der Normaldruck mit $p_0=\SI{1030}{mbar}$.
