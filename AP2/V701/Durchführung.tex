Bei dem verwendetem $\alpha$-Strahler handelt es sich um ein $Am$-Präperat welches zu einem Neptunium-Teilchen ($Np$) und einem Helium-Teilchen ($He$) zerfällt.
Der Zerfall kann über verschiedene Zwischenstufen laufen, sodass die $He$-Atome unterschiedliche Anfangsenergien besitzen können.
\begin{align*}
  {}^{241}_{95}Am \Rightarrow {}^{237}_{93}Np+{}^4_2He^{++}
\end{align*}
Die Halbwertszeit liegt bei $T_{\frac{1}{2}}=458a$
Der $\alpha$-Strahler ist an einem verschiebbarem Halter in einem evakuierbarem Glaszylinder angebracht. Die Detektion der Energie der Heliumatomen wird über ein
Halbleiter-Sperrschichtzähler durchgeführt. In einer elektronenverarmten Zone werden von den einfallenden Ionen Elektronen-Lochpaare gebildet, welche als Strompuls messbar sind.\\
Die Stromimpulse werden von einem Vorverstärker verstärkt und von dem Vielkanalanalysator in Abhängigkeit von der Pulshöhe analysiert.
Am Vielkanalanalysator kann vor der Messung die Diskreminatorschwellen eingestellt werden.
Die Pulshöhe ist in Abhängigkeit der Energie der $\alpha$-Teilchen und kann in einem Histogramm aufgetragen werden.
