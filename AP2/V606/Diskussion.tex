Die Werte der Suszeptibilität unterliegen verschiedenen Fehlerquellen.
Initial ist auffällig, dass das Amperemeter auch ohne angeschlossene Spannungsquelle stark ausschlägt.
Während der Messung gibt es weitere starke Schwankungen.
Desweiteren können parallave Fehler und Schwankungen im Stromnetz nicht ausgeschlossen werden.
Außerdem wird versehentlich vor dem Messungsbeginn der Parallelwiderstand verändert, sodass nur die Messung für Neodym auswertbar wird.
Zudem filtert der Bandpassfilter nicht alle Frequenzen und die Messung der Filterkurve wurde mit der Güte $Q=10$ statt $Q=100$ durchgeführt.
Weiterhin hat der Funktionsgenerator einen Wackelkontakt.
\\Als theoretischer Wert der Suszeptibilität ergibt sich
\begin{equation*}
  \chi_{theo}=\SI{0.003522}{}.
\end{equation*}
Der errechnete Messwert über die Brückenspannung berechnet sich zu
\begin{equation*}
  \chi_{U}=\SI{0.00301}{}.
\end{equation*}
Die Suszeptibilität über die Widerstände berechnet, beträgt
\begin{equation*}
  \chi_{R}=\SI{0.00215}{}.
\end{equation*}
Die relative Abweichung $f$ wird allgemein über
\begin{equation*}
  f=\frac{x_{exp}-x_{theo}}{x_{theo}}
\end{equation*}
berechnet.
Die Abweichung zwischen dem über die Spannung berechneten Wert $\chi_{U}$ und dem theoretischen Wert beträgt dementsprechend $f_{U}=\SI{14.54}{\%}$.
Der über die Widerstände errechnete Wert weicht um $f_{R}=\SI{38.96}{\%}$ von dem Theoriewert ab.
Insgesamt lässt sich sagen, dass die Erwartungen dennoch erfüllt wurden, denn die Suszeptibilitäten liegen in einer passenden Größenordnung.
