Initial lässt sich sagen, dass der Versuch annähernd problemfrei und wie erwartet verläuft.
\\Der Braggwinkel wird als
\begin{equation*}
  \theta_{Z, exp}=\SI{28}{°}
\end{equation*}
bestimmt.
Der vom Versuch gegebene Theoriewert liegt bei
\begin{equation*}
  \theta_{Z, theo}=\SI{28}{°}.
\end{equation*}
Die relative Abweichung $f$ in Prozent wird allgemein durch
\begin{equation}
  f= \frac{x_{exp}-x_{theo}}{x_{theo}}\cdot 100
  \label{eqn:abw}
\end{equation}
berechnet und ergibt sich hier zu $f_{Bragg}= \SI{0}{\%}$.
\\Bei der Messung des Emissionsspektrums der Kupfer-Röntgenröhre ergeben sich die Winkel $\theta$ der $K_{\alpha}-$ und $K_{\beta}-$Linien zu
\begin{align*}
  \theta_{\alpha, exp}  =& \SI{22.4}{°}\\
  \theta_{\beta, exp}   =& \SI{20.0}{°}.
\end{align*}
Die zugehörigen Literaturwerte \cite{2} aus der Vorbereitungsaufgabe sind:
\begin{align*}
  \theta_{\alpha, theo} =&\SI{22.32}{°}\\
  \theta_{\beta, theo}  =&\SI{20.21}{°}.
\end{align*}
Die relativen Abweichungen ergeben sich nach Gleichung \eqref{eqn:abw} zu:
\begin{align*}
  f_{\theta_{\alpha}} =&  \SI{0.36}{\%} \\
  f_{\theta_{\beta}}  =&  \SI{1.04}{\%}.
\end{align*}
Aus Abbildung \ref{fig:emiss} lässt sich der minimale Winkel $\theta_{min}=\SI{4.8}{°}$ ablesen.
Daraus berechnet sich mit Gleichung \eqref{eqn:winkel} die zugehörige gemessene Wellenlänge
\begin{equation*}
  \lambda_{min, exp}=\SI{3.37e-11}{m}.
\end{equation*}
Mit Gleichung \eqref{eqn:lambdamin} ergibt sich der theoretische Wert der minimalen Wellenlänge:
\begin{equation*}
  \lambda_{min, theo}=\SI{3.54e-11}{m}.
\end{equation*}
Die relative Abweichung $f_{\lambda}$ der minimalen Wellenlänge berechnet sich nach Gleichung \eqref{eqn:abw} zu $f_{\lambda}=\SI{4.80}{\%}$.
Die Halbwertsbreiten der Peaks in Abbildung \ref{fig:emiss} berechnen sich zu:
\begin{align*}
  \Delta \theta_{\alpha} &= \theta_{2}-\theta_{1} = \SI{20.3}{°}-\SI{19.7}{°} = \SI{0.6}{°}\\
  \Delta \theta_{\beta}  &= \theta_{2}-\theta_{1} = \SI{22.6}{°}-\SI{22.2}{°} = \SI{0.4}{°}.
\end{align*}
Die Abschirmkonstante $\sigma_{K}$ wird zu
\begin{equation*}
  \sigma_{K, exp}= \SI{20.76}{}
\end{equation*}
bestimmt.
Als Literaturwert wird
\begin{equation*}
  \sigma_{K, theo}= \SI{21.00}{}
\end{equation*}
verwendet.
Die Werte weichen um $f_{\sigma_{K}}= \SI{1.14}{\%}$ voneinander ab.
\\In der Absorptionspektroskopie werden die drei Elemente Brom, Strontium und Zirkonium als Absorber mit $Z<50$ vor das Geiger-Müller-Zählrohr geschraubt.
Aus ihren Ergebnissen wird die Rydbergenergie $E_{Ry}$ bestimmt.
Danach folgt ein Absorber aus Quecksilber mit $Z>50$.
\\Für den Brom-Absorber ergeben sich die gemessene Absorptionsenergie $E$ und der gemessene Abschirmkoeffizient $\sigma_{K}$ zu:
\begin{align*}
  E_{Br-K, exp}       =& \SI{14.221}{keV} \\
  \sigma_{Br-K, exp}  =& \SI{2.6629}{}
\end{align*}
Die Literaturwerte lauten:
\begin{align*}
  E_{Br-K, theo}      =& \SI{13.47}{keV}\\
  \sigma_{Br-K, theo} =& \SI{3.53}{}.
\end{align*}
Die relativen Abweichungen ergeben sich zu:
\begin{align*}
  f_{E_{Br-K}}      =&  \SI{5.57}{\%}\\
  f_{\sigma_{Br-K}} =&  \SI{24.56}{\%}.
\end{align*}
\\Der Strontium-Absorber führt zu folgenden Werten:
\begin{align*}
  E_{Sr-K, exp}       =& \SI{17.051}{keV} \\
  \sigma_{Sr-K, exp}  =& \SI{2.591}{}
\end{align*}
Als Literaturwerte werden
\begin{align*}
  E_{Sr-K, theo}      =& \SI{16.10}{keV}\\
  \sigma_{Sr-K, theo} =& \SI{3.59}{}
\end{align*}
verwendet.
Die relativen Abweichungen werden errechnet:
\begin{align*}
  f_{E_{Sr-K}}      =&  \SI{5.91}{\%}\\
  f_{\sigma_{Sr-K}} =&  \SI{27.83}{\%}.
\end{align*}
\\Mit dem Zirkonium-Absorber werden die folgenden Werte gemessen:
\begin{align*}
  E_{Zr-K, exp}       =& \SI{19.046}{keV} \\
  \sigma_{Zr-K, exp}  =& \SI{2.576}{}.
\end{align*}
Die theoretischen Werte ergeben sich zu:
\begin{align*}
  E_{Zr-K, theo}      =& \SI{17.99}{keV}\\
  \sigma_{Zr-K, theo} =& \SI{3.63}{}.
\end{align*}
Damit errechnen sich die relativen Abweichungen zu:
\begin{align*}
  f_{E_{Zr-K}}      =&  \SI{5.87}{\%}\\
  f_{\sigma_{Zr-K}} =&  \SI{29.04}{\%}.
\end{align*}
\\Die Rydbergenergie $E_{Ry}$ wird über die K-Kanten der Brom-, Strontium- und Zirkonium-Absorber zu
\begin{equation*}
  E_{Ry, exp}=\SI{14.085 \pm 0.127}{eV}
\end{equation*}
berechnet.
Als Literaturwert wird
\begin{equation*}
  E_{Ry, theo}= \SI{13.605}{eV}
\end{equation*}
verwendet.
Die Werte weichen um $f_{E_{Ry}}= \SI{3.53}{\%}$ voneinander ab.
\\Obwohl nur drei Absorber zur Berechnung der Rydbergenergie verwendet werden können, ist der Wert erstaunlich genau.
\\Der Quecksilber-Absorber wird verwendet um die L-Kanten eines Elements mit $Z>50$ zu messen.
Zunächst wird ein zu kleiner Messbereich gewählt und es können keine L-Kanten erkannt werden.
In der zweiten Messung mit dem Quecksilber-Absorber kommt es zu den folgenden Werten:
\begin{align*}
  E_{Hg-L, exp}       =& \SI{14.804}{keV}\\
  \sigma_{Hg-L, exp}  =& \SI{47}{}.
\end{align*}
Als theoretische Werte ergeben sich:
\begin{align*}
  E_{Hg-L, theo}       =& \SI{14.84}{keV}\\
  \sigma_{Hg-L, theo}  =& \SI{4.44}{}.
\end{align*}
Die relativen Abweichungen liegen damit bei
\begin{align*}
  f_{E_{Hg-L, exp}}       =& \SI{0.24}{\%}\\
  f_{\sigma_{Hg-L, exp}}  =& \SI{0.21}{\%}.
\end{align*}

\FloatBarrier
