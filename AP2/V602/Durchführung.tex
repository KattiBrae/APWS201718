\subsection{Vorbereitungsaufgabe}
Es sollen die Energien $E$ der Cu-$K_{\alpha}$-Linie und der Cu-$K_{\beta}$-Linie rechechiert \cite{3} und der Bragg-Winkel $\theta$ daraus ausgerechnet werden (Gl. \eqref{eqn:winkel}).
\begin{align*}
  E_{Cu-K_{\alpha}}=\SI{8.04}{keV} && \theta=\SI{22.32}{°}\\
  E_{Cu-K_{\beta}}=\SI{8.91}{keV}  && \theta=\SI{20.21}{°}
\end{align*}
Desweiteren wird die Abschirmkonstante $\sigma_{K}$ des Übergangs von $E_{Cu-K_{\beta}}$ zu $E_{Cu-K_{\alpha}}$ mit der Ordnungszahl $Z_{Cu}=\SI{29}{}$ \cite{3} daraus ausgerechnet (Gl. \eqref{eqn:E}):
\begin{align}
  \sigma_{K_{\beta}} &=\SI{3.41}{}\\
  \sigma_{K_{\alpha}} &=\SI{13.01}{}
\end{align}
Außerdem werden die Kernladungszahl $Z$ und die Energie der K-Linie $E$ für die Elemente Brom, Strontium und Zirkonium recherchiert \cite{2}.
Daraus werden der Bragg-Winkel $\theta$ (Gl. \eqref{eqn:winkel}) und der Abschirmkoeffizient $\sigma$ (Gl. \eqref{eqn:E})berechnet.
\begin{align*}
  Brom      &(Br)&  Z=35 & & E_{Br-K}=\SI{13.47}{keV} & & \theta_{Br-K}= \SI{13.21}{°} & & \sigma_{Br-K}=\SI{3.53}{}& \\
  Strontium &(Sr)&  Z=38 & & E_{Sr-K}=\SI{16.10}{keV} & & \theta_{Sr-K}= \SI{11.02}{°} & & \sigma_{Sr-K}=\SI{3.59}{}& \\
  Zirkonium &(Zr)&  Z=40 & & E_{Zr-K}=\SI{17.99}{keV} & & \theta_{Zr-K}= \SI{9.85}{°}  & & \sigma_{Zr-K}=\SI{3.63}{}&
\end{align*}
Desweiteren wird für das Element Quecksilber die Kernladungszahl $Z$ und die Energien der L-Linien $E$ recherchiert \cite{2} um daraus den Bragg-Winkel $\theta$ und den Abschirmkoeffizienten $\sigma$ zu berechnen.
\begin{align*}
  &Quecksilber        &    Hg \\
  &Z                  &=   80 \\
  &\theta_{Hg-L}      &=   \SI{11.97}{°} \\
  &\sigma_{L}         &=   \SI{4.44}{}\\
  &E_{Hg-L}           &=   \SI{14.84}{keV}\\
  &E_{Hg-L_{\alpha}}  &=   \SI{9.99}{keV}\\
  &E_{Hg-L_{\beta}}   &=   \SI{11.82}{keV}
\end{align*}
\subsection{Aufbau}
Die Messapparatur besteht aus einer Röntgenröhre, einem Geiger-Müller-Zähler, einem Braggkristall und einem Rechner zur Bedienung des Messprogramms.
Das Licht der Röntgenröhre strahlt auf den drehbaren Braggkristall.
Von dort fällt das Licht in den Geiger-Müller-Zähler.
Mit dem Rechner können der Kristallwinkel, der Zählrohrwinkel und die Messintervalle reguliert werden.
Die Beschleunigungsspannung wird auf $U_{B}=\SI{35}{kV}$ gestellt und der Strom auf $I=\SI{1}{mA}$.
\subsection{Verifizierung der Bragg-Bedingung}
Zur Verifizierung der Bragg-Bedingung wird der Modus "fester Kristallwinkel" angewählt und der Kristallwinkel wird auf $\theta=\SI{14}{°}$ eingestellt.
Der Zählrohrwinkel wird auf $\SI{26}{°} \le \alpha_{Z} \le \SI{30}{°}$ festgelegt.
Die Messintervalle sind $\Delta \alpha=\SI{0.1}{°}$ groß.
Als Zeit pro Messung wird $\Delta t=\SI{5}{s}$ gewählt.
Die Messung wird ausgeführt, der Graph und die Messwerte werden ausgedruckt.
\subsection{Emissionsspektrum}
Das Emissionsspektrum der Röntgenröhre wird im "2:1-Koppelmodus" gemessen.
Der Kristallwinkelbereich wird als $\SI{4}{°} \le \theta \le \SI{26}{°}$ und die Messintervalle auf $\Delta \alpha = \SI{0.2}{°}$ eingestellt.
Die Messzeit beträgt $\Delta t=\SI{5}{s}$.
Anschließend wird gemessen und die Messwerte und der Graph werden ausgedruckt.
\subsection{Absorptionsspektren}
Das Fenster an der Messapparatur wird geöffnet und es wird ein Zirkonium-Absorber (Z=40) am Geiger-Müller-Zählrohr arritiert.
Die Messung wird im "2:1-Koppelmodus" gemacht.
Als Messbereich wird $\SI{7.85}{°} \le \alpha \le \SI{11.85}{°}$ und die Messzeit als $\Delta t=\SI{20}{s}$ gewählt.
Die Messintervalle betragen $\Delta \alpha = \SI{0.1}{°}$.
\\Anschließend wird eine analoge Messung mit einem Brom-Absorber (Z=35) mit dem Messbereich $\SI{11.21}{°} \le \alpha \le \SI{15.21}{°}$ gemacht.
\\Der dritte Absorber ist ein Strontium-Absorber.
Der Messbereich wird auf $\SI{9.02}{°} \le \alpha \le \SI{13.02}{°}$ eingestellt.
\\Für die letzte Messung wird ein Quecksilber-Absorber (Z=80) eingebaut.
Der Messbereich beträgt $\SI{10}{°} \le \alpha \le \SI{16}{°}$.
\\Für alle Messungen werden sowohl der Graph wie auch die Messwerte ausgedruckt.
\FloatBarrier
