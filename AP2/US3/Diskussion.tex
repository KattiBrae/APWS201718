Initial lässt sich sagen, dass der Versuch entstprechend der Erwartungen verlaufen ist.
Im ersten Versuchsteil sinkt die Flussgeschwindigkeit mit wachsendem Radius.
\\Im zweiten Versuchsteil ist in den Abbildungen \ref{fig:40v} und \ref{fig:65v} ein deutliches parabolisches Geschwindigkeitsprofil zu erkennen.
Der erneute Anstieg der Messwerte ist auf das Eintreten des Schalls in die rückseitige Wand des Rohrs zurückzuführen.
In den Abbildungen \ref{fig:40i} und \ref{fig:65i} sind die Streuintensitäten gegen die Eindringtiefe aufgetragen.
Der Kurvenverlauf ist invers zum Geschwindigkeitsprofil, was durch das Geschwindigkeitsprofil zu erklären ist.
Bei höherer Geschwindigkeit wird die Intensität stärker gestreut und es kommt weniger im Schallkopf an.
Die experimentellen mittleren Flussgeschwindigkeiten werden als
\begin{align*}
  \bar{v}_{3,4 l/min}= & \SI{0.658}{\frac{m}{s}} \\
  \bar{v}_{4,8 l/min}= & \SI{1.005}{\frac{m}{s}} \\
\end{align*}
gemessen.
Die theoretischen Werte werden berechnet zu:
\begin{align*}
  v_{\text{theo, 3,4 l/min}}=& \SI{0,722}{\frac{m}{s}} \\
  v_{\text{theo, 4,8 l/min}}=& \SI{1,019}{\frac{m}{s}}. \\
\end{align*}
Ein relativer Fehler $f$ berechnet sich durch
\begin{align*}
  f=\frac{x_{\text{exp}}-x_{\text{theo}}}{x_{\text{theo}}}.
\end{align*}
Die Abweichungen der theoretischen Werte und der gemittelten gemessenen Werte bei den verschiedenen Radien und Flussgeschwindigkeiten berechnen sich zu:
\begin{align*}
  f_{3,4 l/min}= & \SI{8.86}{\%} \\
  f_{4,8 l/min}= & \SI{1.37}{\%}\\
\end{align*}
Die Abweichungen in den Kurvenverläufen sind unter anderem durch fehlerhaftes Ablesen der Messwerte zu erklären.
Die Messwerte schwanken stark während des Ablesens.
