Bei der Sonographie werden Schallwellen zwischen $\SI{20}{kHz}$ bis $\SI{1}{GHz}$ verwendet.
Dabei handelt es sich bei Schall um eine longitudinale Welle der Form:
\begin{align*}
  p(x, t)= p_{\text{0}}-v_{\text{0}}Z \cos{(\omega t -k x)}.
\end{align*}
$Z$ beschreibt dabei die akustische Impedanz $Z=c \cdot \rho$, also das Produkt aus Schallgeschwindigkeit $c$ und der Dichte $\rho$ des durchschallten Mediums.
Näheres kann in der Theorie zu Versuch US2 nachgelesen werden.\\
Bewegen sich Schallquelle und Beobachter relativ zu einander, kommt es zu einer Frequenzänderung, die als der Doppler-Effekt bekannt ist.
Die Frequenz $\nu_0$ wird zu Frequenz $\nu_{\text{kl}}$ erhöht wenn sich die Quelle auf den Beobachter zu bewegt.
Entfernt sich die Quelle vom Beobachter, wird die Frequenz $\nu_{\text{gr}}$ wahrgenommen.
\begin{align*}
  \nu_{\text{kl/gr}}=\frac{\nu_0}{1\mp\frac{v}{c}}
\end{align*}
Wird nun der Beobachter auf die Quelle zu bewegt, wird die Frequenz $\nu_0$ zu Frequenz $\nu_{\text{h}}$ verschoben.
Bei steigender Entfernung wird die Frequenz kleiner zu dem Wert $\nu_{\text{n}}$.
\begin{align*}
  \nu_{\text{h/n}}=\nu_0\cdot\left(1\pm\frac{v}{c}\right)
\end{align*}
Über die Frequenzverschiebung kann eine Aussage zur Geschwindigkeit gemacht werden.
Da die Quelle im selben Winkel zum bewegten Objekt steht wie der Empfänger, gilt für den Doppler-Effekt folgender Zusammenhang:
\begin{align}
  \Delta \nu = 2 \nu_0 \frac{v}{c}  \text{cos} \alpha
  \label{eqn:v}
\end{align}
\\ Die Schallwellen werden häufig durch den Piezo-elektrischen Effekt erzeugt.
Dabei wird ein Piezokristall durch ein wechselndes elektrisches Feld in Schwingung gebracht.
Diese Wellen können als Ultraschall für die Untersuchung genutzt werden.
Wird die Resonanzfrequenz des Kristalls erreicht, entstehen durch die Resonanzüberhöhung energiereiche Schallwellen.
Andersherum kann ein ruhender Piezokristall durch eine Schallwelle in Schwingung gebracht werden,
die dann als Spannung in das elektrische Feld eingehen.
Damit kann der Piezo-elektrische Effekt sowohl für den Ultraschallsender und den Ultraschallempfänger verwendet werden.
