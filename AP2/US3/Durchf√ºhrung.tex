%Schallgeschwindigkeiten
%Innendurchmesser der Rohre
%
Die Messaperatur besteht aus einem Schallkopf($f=\SI{2}{MHz}$), einem Ultraschallgerät und einem Rechner, der für die Datenaufnahme und die Datenanalyse zuständig ist.
Untersucht wird die Strömung einer viskosen Flüssigkeit in einem Rohrsystem. Eine Zentrifugalpumpe sorgt für einen konstanten Volumenstrom.
In dem Rohrsystem gibt es drei zu messende Strömungsrohre mit unterschiedlichen Innendurchmessern. Desweiteren werden für das Ankoppeln der Ultraschallsonde an die Rohre Dopplerprismen verwendet.
Die Dopplerprismen garantieren einen reproduzierbaren Einstrahlwinkel.  Die Entfernung zwischen Sonde und Rohr ist für alle Einstellungen gleich.
Der Dopplerwinkel $\alpha$ ist abhängig von dem Brechungsgesetz und der Schallgeschwindigkeit $c_\text{L}$ des Dopplermediums und der Schallgeschwindigkeit $c_\text{P}$ im Prismenmaterial.
\begin{align}
  \alpha=90°-\text{arcsin}\left(\text{sin}\theta\cdot \frac{C_L}{C_P}\right)
  \label{eqn:alpha}
\end{align}
\\Für 5 Geschwindigkeiten wird an den drei Rohren die Ultraschallmessung, zur Bestimmung der Frequenzverscheibung $\Delta\nu$, für alle drei Winkel durchgeführt.
Mit Hilfe von $\Delta\nu$ kann die Strömungsgeschwindigkeit ermittelt werden.
\\Nun wird das Strömungsprofil der Dopplerflüssigkeit untersucht. Dazu wird am Strömungsrohr, mit dem Innendurchmesser von $\SI{10}{mm}$, unter einem 15° Winkel, für unterschiedliche Pumpleistungen, eine Untersuchung mit variabler Meßtiefe durchgeführt.
Es wird die Strömungsgeschwindigkeit und die Streuintensität gemessen.
\begin{align*}
  \text{Strömungsrohre}:&& \text{Innendurchmesser}\\
               &&    \SI{7}{mm}\\
               &&    \SI{10}{mm}\\
               &&    \SI{16}{mm}\\
\end{align*}
\begin{align*}
  \text{Dopplerprisma}:&&c_\text{P}=\SI{2700}{\frac{m}{s}}&&\text{Schallgeschwindigkeit}\\
                       &&l=\SI{30,7}{mm}&&\text{Länge der Vorlaufstrecke}\\
  \text{Dopplerflüssigkeit}: && c_\text{L}=\SI{1800}{\frac{m}{s}} && \text{Schallgeschwindigkeit}\\
\end{align*}
